\documentclass[12pt,a4paper]{scrartcl} 
\usepackage[utf8]{inputenc}
\usepackage[english,russian]{babel}
\usepackage{indentfirst}
\usepackage{misccorr}
\usepackage{graphicx}
\usepackage{amsmath}
\begin{document}
	\begin{titlepage}
		\begin{center}
			\large
			МИНИСТЕРСТВО НАУКИ И ВЫСШЕГО ОБРАЗОВАНИЯ РОССИЙСКОЙ ФЕДЕРАЦИИ
			
			Федеральное государственное бюджетное образовательное учреждение высшего образования
			
			\textbf{АДЫГЕЙСКИЙ ГОСУДАРСТВЕННЫЙ УНИВЕРСИТЕТ}
			\vspace{0.25cm}
			
			Инженерно-физический факультет
			
			Кафедра автоматизированных систем обработки информации и управления
			\vfill

			\vfill
			
			\textsc{Отчет по практике}\\[5mm]
			
			{\LARGE \textit{Реализовать метод половинного деления для нахождения корня функции.}}
			\bigskip
			
			1 курс, группа 1УТС
		\end{center}
		\vfill
		
		\newlength{\ML}
		\settowidth{\ML}{«\underline{\hspace{0.7cm}}» \underline{\hspace{2cm}}}
		\hfill\begin{minipage}{0.5\textwidth}
			Выполнил:\\
			\underline{\hspace{\ML}} А.\,В. Смирнов\\
			«\underline{\hspace{0.7cm}}» \underline{\hspace{2cm}} 2021 г.
		\end{minipage}%
		\bigskip
		
		\hfill\begin{minipage}{0.5\textwidth}
			Руководитель:\\
			\underline{\hspace{\ML}} С.\,В. Теплоухов\\
			«\underline{\hspace{0.7cm}}» \underline{\hspace{2cm}} 2021 г.
		\end{minipage}%
		\vfill
		
		\begin{center}
			Майкоп, 2021 г.
		\end{center}
	\end{titlepage}
\section{Текстовая формулировка задачи}
\label{sec:exp:code}
\begin{verbatim}
Реализовать метод половинного деления для нахождения корня функции.
Алгоритм:
1)	Найдём значение x в середине отрезка: xM=(xL+xR)/2
2)	Вычислим значение функции f(xM) в середине отрезка  xM:
Если f(xM)=0 или, в действительных вычислениях,|f(xM)| <= Ef(x),где Ef(x) — 
заданная точность по оси y, то корень найден.
Иначе а(xM) != 0 или, в действительных вычислениях,|f(xM)| > Ef(x), то
разобьём отрезок [xL,xR] на два равных отрезка: [xL,xM] и [xM,xR].
3)	Теперь найдём новый отрезок, на котором функция меняет знак:
Если значения функции на концах отрезка имеют противоположные знаки на левом
отрезке, f(xL)*f(xM)<0,то, соответственно, корень находится внутри левого
отрезка [xL,xM]. Тогда возьмём левый отрезок присвоением xR=xM, и повторим
описанную процедуру до достижения требуемой точности Ef(x) по оси y.
Иначе значения функции на концах отрезка имеют противоположные знаки на
правом отрезке, f(xM)*f(xR)<0, то, соответственно, корень находится внутри
правого отрезка [xM,xR]. Тогда возьмём правый отрезок присвоением xL=xM,
и повторим описанную процедуру до достижения требуемой точности Ef(x) по
оси y.
\end{verbatim}
\newpage
\section{Код приложения}
\label{sec:exp:code}
\begin{verbatim}
#include <iostream>
using namespace std;
double F(double x){
return x*x-10*x+25;}
double Root(double (*f)(double),double a,double b,double eps){
double c;
while((b-a)/2>eps){
 c=(a+b)/2;
 if((f(a)*f(c))>0) a=c;
 else b=c;
}
return c;}
int main(){
double a,b,eps,x;
cout << "interval: ";
cin >> a;
cin >> b;
if(F(a)*F(b)>0){
 cout << "Wrong interval!\n";
 return 0;}
cout << "eror: ";
cin >> eps;
x=Root(F,a,b,eps);
cout << "x = " << x << endl;
return 0;
}
\end{verbatim}
\newpage
\section{График}
\label{sec:picexample}
\begin{figure}[h]
	\includegraphics[width=0.9\textwidth]{график.png}
	\caption{График функции.}\label{fig:par}
\end{figure}
\newpage
\section{Скриншот программы}
\begin{figure}[h]
	\includegraphics[width=0.9\textwidth]{код.png}
	\caption{Пример работы программы.}\label{fig:par}
\end{figure}
\newpage
\begin{thebibliography}{9}
\bibitem{Knuth-2003}Кнут Д.Э. Всё про \TeX. \newblock --- Москва: Изд. Вильямс, 2003 г. 550~с.
\bibitem{Lvovsky-2003}Львовский С.М. Набор и верстка в системе \LaTeX{}. \newblock --- 3-е издание, исправленное и дополненное, 2003 г.
\bibitem{Voroncov-2005}Воронцов К.В. \LaTeX{} в примерах. 2005 г.
\end{thebibliography}
\end{document}